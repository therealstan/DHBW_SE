\documentclass[]{scrartcl}

\usepackage[sort, square]{natbib}
\usepackage{hyperref}
\usepackage{amsmath}
\usepackage{graphicx}
\usepackage{amsmath}
\usepackage{amssymb}
\usepackage{amsfonts}
% sichert die richtige Darstellung der deutschen Sprache
\usepackage[ngerman]{babel}
\usepackage[utf8]{inputenc}
\usepackage[T1]{fontenc}
% % % % % % % %
\usepackage{multirow}
\usepackage{geometry}
\usepackage{setspace}
\usepackage{siunitx}
\usepackage{textcomp}
\usepackage{colortbl}
\usepackage{extarrows}

\newcommand\abb[1]{\textbf{\underline{#1}}}
% Startet neue Seite pro \section{} Befehl
\usepackage{titlesec}
\newcommand{\sectionbreak}{\clearpage}
% % % % % % % %

\usepackage{pgf}
\usepackage{color}
\definecolor{mygray}{rgb}{0.5,0.5,0.5}
\definecolor{LimeGreen}{HTML}{9EFD38}
\usepackage{cancel}
\usepackage{url}
\usepackage{epstopdf}
\sisetup{ output-decimal-marker = {,}} % Decimalzahlen sind durch Komma getrennt

% Ein Caption für ein Listing wird definiert
\usepackage{caption}
\usepackage{listings}

\DeclareCaptionFont{white}{ \color{white} }
\DeclareCaptionFormat{listing}{
	\colorbox[cmyk]{0.43, 0.35, 0.35,0.01 }{
		\parbox{\textwidth}{\hspace{15pt}#1#2#3}
	}
}
\captionsetup[lstlisting]{ format=listing, labelfont=white, textfont=white, singlelinecheck=false, margin=0pt, font={bf,footnotesize} }
% % % % % % % % % % % % % % % % % % % % % % % % % % % % % % % % % % % %


\geometry{outer=30mm,
inner=30mm,
top=20mm,
bottom=30mm}

\newcommand\grad{^{\circ}}
% 
%
% Unterstreichen und Umrahmen der Formeln in align-Umgebung
\newcommand\ul[2][]{%Befehl zum Unterstreichen von #2 mit eventueller Größenabstimmung mit #1 
 \underline{#2\vphantom{#1}}}%wenn #1 angegeben, wird Tiefe der Linie angepaßt 

\newcommand\ulalign[2]{\ul[#2]{#1}&\ul[#1]{\;=#2}}%einfaches Unterstreichen einer align-Gl. mit &= 
\newcommand\dulalign[2]{\underline{\ul[#2]{#1}}&\underline{\ul[#1]{\;=#2}}}%doppeltes Unterstreichen einer align-Gl. mit &= 


\usepackage{tikz} 
\usetikzlibrary{fit} 

\newcommand\raalign[2]{%Umrahmen einer align-Gl. mit &= 
   \tikz[overlay,every node/.style={inner sep=0pt,outer sep=0pt}]% 
      \node[anchor=base west](g){\phantom{$\displaystyle #1\null=\null#2$}}% 
         node[draw=gray!70!black,fill=gray!20,fit=(g),inner sep=5pt]{};% 
   #1&=#2} 
  
% 
%
% 


\usepackage{listings} % Quellcode im Text
\lstset{language=JAVA, % Programmiersprache
		tabsize=1, 	  % Länge des Tab Zeichens
        basicstyle=\ttfamily\color{mygray}\small, % style des Codes
        keywordstyle=\ttfamily\color{blue},
        identifierstyle=\color{mygray},
        commentstyle=\color{red},
        stringstyle=\ttfamily\color{black},
        showstringspaces=false,
        numbers=left, 
        numberstyle=\tiny, 
        stepnumber=1, 
        numbersep=5pt,
        breaklines=true,
        literate=%   
            {Ö}{{\"O}}1
            {Ä}{{\"A}}1
            {Ü}{{\"U}}1
            {ß}{{\ss}}1
            {ü}{{\"u}}1
            {ä}{{\"a}}1
            {ö}{{\"o}}1
            {~}{{\textasciitilde}}1}

\newcommand{\HRule}{\rule{\linewidth}{0.5mm}}

\usepackage{expdlist}

\usepackage[tocflat]{tocstyle}   
\usetocstyle{allwithdot}
