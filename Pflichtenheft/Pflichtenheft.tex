\documentclass[]{scrartcl}

\usepackage[sort, square]{natbib}
\usepackage{hyperref}
\usepackage{amsmath}
\usepackage{graphicx}
\usepackage{amsmath}
\usepackage{amssymb}
\usepackage{amsfonts}
% sichert die richtige Darstellung der deutschen Sprache
\usepackage[ngerman]{babel}
\usepackage[utf8]{inputenc}
\usepackage[T1]{fontenc}
% % % % % % % %
\usepackage{multirow}
\usepackage{geometry}
\usepackage{setspace}
\usepackage{siunitx}
\usepackage{textcomp}
\usepackage{colortbl}
\usepackage{extarrows}

\newcommand\abb[1]{\textbf{\underline{#1}}}
% Startet neue Seite pro \section{} Befehl
\usepackage{titlesec}
\newcommand{\sectionbreak}{\clearpage}
% % % % % % % %

\usepackage{pgf}
\usepackage{color}
\definecolor{mygray}{rgb}{0.5,0.5,0.5}
\definecolor{LimeGreen}{HTML}{9EFD38}
\usepackage{cancel}
\usepackage{url}
\usepackage{epstopdf}
\sisetup{ output-decimal-marker = {,}} % Decimalzahlen sind durch Komma getrennt

% Ein Caption für ein Listing wird definiert
\usepackage{caption}
\usepackage{listings}

\DeclareCaptionFont{white}{ \color{white} }
\DeclareCaptionFormat{listing}{
	\colorbox[cmyk]{0.43, 0.35, 0.35,0.01 }{
		\parbox{\textwidth}{\hspace{15pt}#1#2#3}
	}
}
\captionsetup[lstlisting]{ format=listing, labelfont=white, textfont=white, singlelinecheck=false, margin=0pt, font={bf,footnotesize} }
% % % % % % % % % % % % % % % % % % % % % % % % % % % % % % % % % % % %


\geometry{outer=30mm,
inner=30mm,
top=20mm,
bottom=30mm}

\newcommand\grad{^{\circ}}
% 
%
% Unterstreichen und Umrahmen der Formeln in align-Umgebung
\newcommand\ul[2][]{%Befehl zum Unterstreichen von #2 mit eventueller Größenabstimmung mit #1 
 \underline{#2\vphantom{#1}}}%wenn #1 angegeben, wird Tiefe der Linie angepaßt 

\newcommand\ulalign[2]{\ul[#2]{#1}&\ul[#1]{\;=#2}}%einfaches Unterstreichen einer align-Gl. mit &= 
\newcommand\dulalign[2]{\underline{\ul[#2]{#1}}&\underline{\ul[#1]{\;=#2}}}%doppeltes Unterstreichen einer align-Gl. mit &= 


\usepackage{tikz} 
\usetikzlibrary{fit} 

\newcommand\raalign[2]{%Umrahmen einer align-Gl. mit &= 
   \tikz[overlay,every node/.style={inner sep=0pt,outer sep=0pt}]% 
      \node[anchor=base west](g){\phantom{$\displaystyle #1\null=\null#2$}}% 
         node[draw=gray!70!black,fill=gray!20,fit=(g),inner sep=5pt]{};% 
   #1&=#2} 
  
% 
%
% 


\usepackage{listings} % Quellcode im Text
\lstset{language=JAVA, % Programmiersprache
		tabsize=1, 	  % Länge des Tab Zeichens
        basicstyle=\ttfamily\color{mygray}\small, % style des Codes
        keywordstyle=\ttfamily\color{blue},
        identifierstyle=\color{mygray},
        commentstyle=\color{red},
        stringstyle=\ttfamily\color{black},
        showstringspaces=false,
        numbers=left, 
        numberstyle=\tiny, 
        stepnumber=1, 
        numbersep=5pt,
        breaklines=true,
        literate=%   
            {Ö}{{\"O}}1
            {Ä}{{\"A}}1
            {Ü}{{\"U}}1
            {ß}{{\ss}}1
            {ü}{{\"u}}1
            {ä}{{\"a}}1
            {ö}{{\"o}}1
            {~}{{\textasciitilde}}1}

\newcommand{\HRule}{\rule{\linewidth}{0.5mm}}

\usepackage{expdlist}

\usepackage[tocflat]{tocstyle}   
\usetocstyle{allwithdot}

\usepackage{float}

\begin{document}
	
\begin{titlepage}
	
	\begin{center}
		
		\textsc{\LARGE Pflichtenheft}\\[2cm]
		\textsc{\large im Studiengang Informationstechnik}\\
		der\\
		\textsc{\large Dualen Hochschule Baden-Württemberg}\\
		\textsc{\large Standort Stuttgart}\\
		\textsc{\large Software-Engineering}\\[3.2cm]
		
		%\includegraphics[width=\textwidth]{./Logo.eps}\\[1cm]

		\HRule \\[0.4cm]
		{\huge \bfseries Bewertungssytem}\\[0.4cm]
		\HRule \\[0.1cm]
		\large von:\\ \textsc{Stanislav Sokol ,Louis Steinkamp, Dominik Zipperle}\\[0.2cm]
		Datum: \today
		\vspace{10.0cm}
			
	\end{center}

\end{titlepage}
	\tableofcontents
	\newpage

	\section{Zielbestimmung}
		\subsection{Musskriterien}
			\begin{itemize}
			\item[-]	Verschiedene Benutzergruppen 
			\item[-]	Benutzer anlegen
			\item[-]	Rechtehierarchien
			\item[-]	Studenten können ihre Noten in dem System einsehen.
			\item[-]	Notenkonvertierungsprofile erstellen und verwalten
			\item[-]	Dozenten weisen einer Veranstaltung Bewertungsschema zu
			\item[-]	Ein Template kann zu einer Veranstaltung erstellt werden.
			\item[-]	Veranstaltungen des Typs (Gruppenarbeit, Einzelarbeit) können erstellt werden.
			\item[-]	Die Festlegung eines Grundscore (Gruppe, einzeln) lässt sich in zwei Skalenbereiche beliebig teilen ($H_2$)
			\item[-]	Score Refining Faktoren können in einem Template frei eingestellt werden (Items) ($R_2S$)
			\item[-]	Skalierbarkeit des Bewertungssystems von Bewertung der Gruppe zu Einzelperson und umgekehrt.
			\item[-]	Zuweisung von beliebig gruppierten Studenten zu einer \item[-]	Veranstaltung durch Dozent oder Prüfer.				
			\end{itemize}

			
		\subsection{Wunschkriterien}
		\begin{itemize}
		\item[-]	Alle Zwischenbewertungen die zur Note führen werden dem Studenten angezeigt.
		\item[-]	Profilverwaltung um standardisierte Bewertungen erzeugen zu können.
		\end{itemize}
		
		\subsection{Abgrenzungskriterien}
		\begin{itemize}
		\item[-]	Bewertungssystem bewertet nur die Veranstaltungen die einer Art bestanden/nicht bestanden Logik unterliegen. Anschließend erfolgt eine konfigurierbare Punkte-Notenkonvertierung.
		\item[-]	Jede erstellte Veranstaltung beinhaltet genau ein Bewertungsschema und führt somit zu einer Note. Die Kombi-nierbarkeit von verschiedenen Einzelnoten ist nicht möglich.
		\end{itemize}
		
	\section{Produkteinsatz}
	
		
		\subsection{Anwendungsbereiche}
		\begin{itemize}
		\item[-]	Planung einer Veranstaltung mit anstehenden Prüfungsleistungen für Lehrveranstaltungsplaner
		\item[-]	Errechnung einer Note durch den Prüfer anhand festgelegter Kriterien.
		\item[-]	Einsicht der Noten durch Studenten, sowie die Leistungsübersicht für die Dozenten und Prüfer.
		\end{itemize}
		
		
		\subsection{Zielgruppen}
		\begin{itemize}
		\item[-]	Lehrveranstaltungsplaner
		\item[-]	Prüfer
		\item[-]	Studenten
		\end{itemize}

		\subsection{Betriebsbedingungen}
		\begin{itemize}
		\item[-]	Das System soll nur einmal angelegt werden und autark funktionieren.
		\item[-]	Klar definierte Benutzergruppen und Rechte
		\item[-]	Datenbankschnittstelle für Benutzerverwaltung
		\item[-]	Datenbank-Umgebung für die Speicherung der Daten
		\item[-]	Archivierung der Daten (Datensicherheit wegen Einsehbarkeit der Daten)
		\item[-]	Zugriff von außen sowohl als Lesen, als auch als Edit möglich. 
		\end{itemize}

		
		
		
		
	\section{Produktübersicht}
	
	\begin{figure}[th!]
	\centering
	\includegraphics[width=\textwidth]{./img/use_case}
	\caption{Darstellung des Gesamtsystem anhand der ausgearbeiteten Use-Cases}
	\label{fig:use_case}
	\end{figure}
	
	\begin{figure}[th!]
	\centering
	\includegraphics[width=\textwidth]{./img/ablauf_1}
	\caption{Prozessdiagramm für Geschäftsprozess Veranstaltungserstellung}
	\label{fig:process1}
	\end{figure}
		
	\begin{figure}[th!]
	\centering
	\includegraphics[width=0.6\textwidth]{./img/ablauf_2}
	\caption{Ablaufdiagramm für den Geschäftsprozess Bewertung}
	\label{fig:process2}
	\end{figure}

	\section{Produktfunktionen}
	\begin{itemize}
	\item[/F10/]   -  Planung einer Bewertungsroutine für eine Veranstaltung. Dies ist nur für die Benutzergruppe „Dozent“ möglich.  Die Planung erfolgt derart, dass zunächst ein Template für die Veranstaltung konfiguriert wird. Ein ferti-ges Template kann nun auf die angelegten Studentengruppen angewandt werden. 
		
			\item[/F11/] - Template für Gruppenprojekt erstellen.
			\item[/F12/] - Template für Einzelprojekt erstellen.
			\item[/F13/] - Refining Kriterien für Aufgabentypus festlegen.	
			\item[/F14/] - Anlegen von Konvertierungsprofilen. 
		\item[/W11/] - Das Template kann abgewandelt werden (z.B. klonen eines Templates, editieren etc.)
			\item[/W12/] - Das Bewertungssystem kann Profile verwalten. 
			\item[/W13/] - Standardisierung der Bewertungen
			
		\item[/F20/] - verwalten von Studenten
			\item[/F22/] - Anlegen von Kurs + Teilnehmer (Gruppe 1.Art)
			\item[/F21/] - Gruppierung von Studenten zur Teamaufgabe (Gruppe 2.Art)
			\item[/F23/] - Anlegen eines  Studenten
		
		\item[/F30/] - verwalten von Prüfer
			\item[/F31/] - Anlegen von Prüfer
			\item[/F32/] - Zuweisung eines Prüfers zu einem Template, Kurs, Student
		
		\item[/F40/] - Ein Prüfer pflegt die Punkte pro Student/Gruppe für eine Veranstaltung in das Bewertungssystem ein. Er arbeitet mit einem durch das Template definierten Bewertungskatalog.
			\item[/F41/] - Rate eintragen
			
		\item[/F50/] - Studenten können Noten einsehen
			\item[/W51/] - Studenten können die Notenentstehung in jeder Einzelheit nachvollziehen (Transparenz)
			\item[/W51/] - Studenten sehen, wie sie im Bezug zum Kurs (anderen Kursen) abgeschnitten haben
		\item[/W60/] - Erstellung eines Notenreports für Einzelpersonen und Gruppen
	\end{itemize}
	
	
	\section{Produktdaten}
	\begin{itemize}
	\item[/D10/] - Studenteninformation
		\item[/D11/] - Kursteilnahmen
		\item[/D12/] - Zugeteilte Prüfer
		\item[/D13/] - Leistungen
		\item[/D14/] - Matrikelnummer
		\item[/D15/]  - Name, Vorname
	\item[/D20/] - Prüfer
		\item[/D21/] -  Kurse
		\item[/D22/] - ID
		\item[/D23/] - Name, Vorname
	\item[/D30/] - Dozenten
		\item[/D31/] - ID
		\item[/D32/] - Name, Vorname
		\item[/D33/] - erstellte Templates
	\item[/D40/] - Veranstaltungen
		\item[/D41/] - Templates für die Bewertungen
		\item[/D42/] - Statistiken über die Leistungen über die Jahre
	\item[/D50/] - Notenkonvertierung
		\item[/D51/] - ID
		\item[/D52/] - Name
		\item[/D53/] - Parameter
	\end{itemize}
	
	\section{Produktleistungen}
	\begin{itemize}
	\item[-]	Die Anwendung (Prototyp) ist nicht zeit- oder  rechenkritisch.
	\item[-]	Die steigenden Benutzerzahlen die zur gleichen Zeit auf das System zugreifen, sollen jedoch keine signifikanten Einflüsse auf die Antwortzeiten des Systems haben. 
	\item[-]	Die Zugriffe von außerhalb, wie z.B. über WWW sind in dem Prototypen nicht vorgesehen.
	\item[-]	Zu speichernde Daten, werden serialisiert gespeichert. (Für eine spätere Implementierung ist der \item[-]	Einsatz einer Datenbank möglich, jedoch für die funktionelle Prototypisierung nicht sofort notwendig.)
	\end{itemize}

	
	\section{Qualitätsanforderungen}
	\begin{table}[ht]
	\caption{Qualitätsmerkmale nach DIN ISO 9126 – siehe Anhang A in T3-4}
	\centering
		\begin{tabular}{|c|c|c|c|c|}
		\hline Produktqualität & sehr gut & gut & normal & nicht relevant \\ 
		\hline \multicolumn{5}{|c|}{Funktionalität}   \\ 
		\hline Angemessenheit &  & \checkmark &  &  \\ 
		\hline Richtigkeit &  & \checkmark  &  &  \\ 
		\hline Interoperabilität &  &\checkmark   &  &  \\ 
		\hline Ordnungsmäßigkeit &  &\checkmark   &  &  \\ 
		\hline Sicherheit &  &\checkmark   &  &  \\ 
		\hline \multicolumn{5}{|c|}{Zuverlässigkeit}   \\ 
		\hline Reife &  &\checkmark   &  &  \\ 
		\hline Fehlertoleranz &  \checkmark &  &  &  \\ 
		\hline Wiederherstellbarkeit & \checkmark  &  &  &  \\ 
		\hline \multicolumn{5}{|c|}{Benutzbarkeit}  \\ 
		\hline Verständlichkeit &  &\checkmark   &  &  \\ 
		\hline Erlernbarkeit &  & \checkmark  &  &  \\ 
		\hline Bedienbarkeit &  & \checkmark  &  &  \\ 
		\hline \multicolumn{5}{|c|}{Effizienz} \\ 
		\hline Zeitverhalten &  &  &\checkmark   &  \\ 
		\hline Verbrauchsverhalten &  &  &\checkmark   &  \\ 
		\hline \multicolumn{5}{|c|}{Änderbarkeit} \\ 
		\hline Analysierbarkeit &  & \checkmark  &  &  \\ 
		\hline Modifizierbarkeit &  & \checkmark  &  &  \\ 
		\hline Stabilität & \checkmark  &  &  &  \\ 
		\hline Prüfbarkeit & \checkmark  &  &  &  \\ 
		\hline \multicolumn{5}{|c|}{Übertragbarkeit}  \\ 
		\hline Anpassbarkeit & \checkmark  &  &  &  \\ 
		\hline Installierbarkeit & \checkmark  &  &  &  \\ 
		\hline Konformität	 &  &  &\checkmark   &  \\ 
		\hline Austauschbarkeit &  &  & \checkmark  &  \\ 
		\hline 
		\end{tabular} 
	\label{tab:quali_anf}
	\end{table}

	
	\section{Benutzeroberfläche}
	
	Die Benutzeroberfläche entspricht den modernen Ansprüchen der Web-Anwendungen und wird mit entsprechend Google-Designrichtlinien erstellt. UI/UX werden an den Zielgruppen getestet. Die intuitive Bedienung ist eine Vo-raussetzung.
	
	\begin{table}[ht]
	\caption{Benutzergruppen und Rechteverteilung}
	\begin{tabular}{|c|c|c|c|c|}
	\hline Benutzergruppe & Lesen & Schreiben & Ändern & Systemanpassungen \\ 
	\hline Administrator & \checkmark & \checkmark & \checkmark & \checkmark \\ 
	\hline Verwalter & \checkmark & \checkmark & \checkmark &  \\ 
	\hline Prüfer & \checkmark & \checkmark &  &  \\ 
	\hline Student & \checkmark &  &  &  \\ 
	\hline 
	\end{tabular} 
	\label{tab:usergroup}
	\end{table}
	
	
	\section{Nichtfunktionale Anforderungen}
	Die Anforderungen entsprechen den Richtlinien der jeweiligen Instanz und werden durch den Administrator ein-malig eingestellt.
	Sicherheitsanforderungen entsprechen den modernen Sicherheitsstandard
	
	\section{Technische Produktumgebung}
	Die technische Landschaft besteht aus einem Server und einem Web-Server, welcher die Anfragen der webba-sierten Clients entgegennimmt. 
		\subsection{Software}
		Wird spezifiziert 
		\subsection{Hardware}
		Sind für dieses Projekt nicht relevant. (prinzipiell ein Leistungsstarker Server )
		\subsection{Orgware}
		Kundenmanagementsystem
		\subsection{Produkt-Schnittstellen}
		Web-Schnittstelle.
	\section{Spezielle Anforderungen an die Entwicklungs-Umgebung}
		\subsection{Software}
		Git Versionsverwaltung, Issue Tracking System, IDE for JAVA Development
		\subsection{Hardware}
		Laptop, Standalone PC
		\subsection{Orgware}
		XP (extreme proramming und agile development)	
		\subsection{Entwicklungs-Schnittstellen}
		WEB-Schnittstelle
	\section{Gliederung in Teilprodukte}
	Der Prototyp besteht aus einem  Veranstaltungskonfigurator, einem Bewertungssystem und einem Modul zum Einsehen der Bewertung.
	Veranstaltungskonfigurator beschränkt sich hierbei auf einen minimalen Funktionsumfang um die mögliche Ar-beitsweise deutlich zu machen. 
	Konfigurieren bedeutet in diesem Zusammenhang die Notenumrechnung mit Parametern zu belegen und zwi-schen unterschiedlichen Bewertungsschemata auswählen zu können. 
	Des Weiteren lassen sich zu einer Veranstaltung einzeln gewichtete Bewertungskritierien hinzufügen. 
	Das Bewertungssystem beinhaltet eine vom Dozent vordefinierte Maske, in die der Prüfer die Ergebnisse der Ver-anstaltung eintragen kann. Anhand der Ergebnisse wird dann eine Note für diese Veranstaltung errechnet.
	Über ein Modul zum Einsehen der Bewertung, können zum einen Dozent und Prüfer die eingegeben Ergebnisse überprüfen, zum anderen die Studenten ihr Abschneiden bei der jeweiligen Veranstaltung überprüfen.
		
	
	\section{Ergänzungen}
	Wird angepasst	
	
	
 \begin{appendix}
  \section{Begriffsdefinitionen}
  \section{Abkürzungen}
  \section{Modelle}
  \section{Qualitätsmerkmale}
  \clearpage
  \section{Aufwandsabschätzung}
  
  	\begin{figure}[ht]
	\centering
	\includegraphics[width=0.9\textwidth]{./files/Aufwand}
	\label{fig:Aufwand}
	\end{figure}

  \end{appendix}

			  
\end{document}