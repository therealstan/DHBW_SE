\documentclass[]{scrartcl}

\usepackage[sort, square]{natbib}
\usepackage{hyperref}
\usepackage{amsmath}
\usepackage{graphicx}
\usepackage{amsmath}
\usepackage{amssymb}
\usepackage{amsfonts}
% sichert die richtige Darstellung der deutschen Sprache
\usepackage[ngerman]{babel}
\usepackage[utf8]{inputenc}
\usepackage[T1]{fontenc}
% % % % % % % %
\usepackage{multirow}
\usepackage{geometry}
\usepackage{setspace}
\usepackage{siunitx}
\usepackage{textcomp}
\usepackage{colortbl}
\usepackage{extarrows}

\newcommand\abb[1]{\textbf{\underline{#1}}}
% Startet neue Seite pro \section{} Befehl
\usepackage{titlesec}
\newcommand{\sectionbreak}{\clearpage}
% % % % % % % %

\usepackage{pgf}
\usepackage{color}
\definecolor{mygray}{rgb}{0.5,0.5,0.5}
\definecolor{LimeGreen}{HTML}{9EFD38}
\usepackage{cancel}
\usepackage{url}
\usepackage{epstopdf}
\sisetup{ output-decimal-marker = {,}} % Decimalzahlen sind durch Komma getrennt

% Ein Caption für ein Listing wird definiert
\usepackage{caption}
\usepackage{listings}

\DeclareCaptionFont{white}{ \color{white} }
\DeclareCaptionFormat{listing}{
	\colorbox[cmyk]{0.43, 0.35, 0.35,0.01 }{
		\parbox{\textwidth}{\hspace{15pt}#1#2#3}
	}
}
\captionsetup[lstlisting]{ format=listing, labelfont=white, textfont=white, singlelinecheck=false, margin=0pt, font={bf,footnotesize} }
% % % % % % % % % % % % % % % % % % % % % % % % % % % % % % % % % % % %


\geometry{outer=30mm,
inner=30mm,
top=20mm,
bottom=30mm}

\newcommand\grad{^{\circ}}
% 
%
% Unterstreichen und Umrahmen der Formeln in align-Umgebung
\newcommand\ul[2][]{%Befehl zum Unterstreichen von #2 mit eventueller Größenabstimmung mit #1 
 \underline{#2\vphantom{#1}}}%wenn #1 angegeben, wird Tiefe der Linie angepaßt 

\newcommand\ulalign[2]{\ul[#2]{#1}&\ul[#1]{\;=#2}}%einfaches Unterstreichen einer align-Gl. mit &= 
\newcommand\dulalign[2]{\underline{\ul[#2]{#1}}&\underline{\ul[#1]{\;=#2}}}%doppeltes Unterstreichen einer align-Gl. mit &= 


\usepackage{tikz} 
\usetikzlibrary{fit} 

\newcommand\raalign[2]{%Umrahmen einer align-Gl. mit &= 
   \tikz[overlay,every node/.style={inner sep=0pt,outer sep=0pt}]% 
      \node[anchor=base west](g){\phantom{$\displaystyle #1\null=\null#2$}}% 
         node[draw=gray!70!black,fill=gray!20,fit=(g),inner sep=5pt]{};% 
   #1&=#2} 
  
% 
%
% 


\usepackage{listings} % Quellcode im Text
\lstset{language=JAVA, % Programmiersprache
		tabsize=1, 	  % Länge des Tab Zeichens
        basicstyle=\ttfamily\color{mygray}\small, % style des Codes
        keywordstyle=\ttfamily\color{blue},
        identifierstyle=\color{mygray},
        commentstyle=\color{red},
        stringstyle=\ttfamily\color{black},
        showstringspaces=false,
        numbers=left, 
        numberstyle=\tiny, 
        stepnumber=1, 
        numbersep=5pt,
        breaklines=true,
        literate=%   
            {Ö}{{\"O}}1
            {Ä}{{\"A}}1
            {Ü}{{\"U}}1
            {ß}{{\ss}}1
            {ü}{{\"u}}1
            {ä}{{\"a}}1
            {ö}{{\"o}}1
            {~}{{\textasciitilde}}1}

\newcommand{\HRule}{\rule{\linewidth}{0.5mm}}

\usepackage{expdlist}

\usepackage[tocflat]{tocstyle}   
\usetocstyle{allwithdot}

\usepackage{float}

\begin{document}
% Titelblatt
	
\begin{titlepage}
	
	\begin{center}
		
		\textsc{\LARGE Pflichtenheft}\\[2cm]
		\textsc{\large im Studiengang Informationstechnik}\\
		der\\
		\textsc{\large Dualen Hochschule Baden-Württemberg}\\
		\textsc{\large Standort Stuttgart}\\
		\textsc{\large Software-Engineering}\\[3.2cm]
		
		%\includegraphics[width=\textwidth]{./Logo.eps}\\[1cm]

		\HRule \\[0.4cm]
		{\huge \bfseries Bewertungssytem}\\[0.4cm]
		\HRule \\[0.1cm]
		\large von:\\ \textsc{Stanislav Sokol ,Louis Steinkamp, Dominik Zipperle}\\[0.2cm]
		Datum: \today
		\vspace{10.0cm}
			
	\end{center}

\end{titlepage}
% % % %
	\tableofcontents
	
% Versionshistorie	
	
	\begin{versionhistory}
		\vhEntry{1.1}{25.09.2014}{\sA}{Formulierung Muss- /Wunschkriterien}
		\vhEntry{1.2}{26.09.2014}{\sA}{Formulierung Musskriterien}
		\vhEntry{1.3}{16.10.2014}{\sA}{Ausarbeitung der Produktfunktionen}
		\vhEntry{1.4}{17.10.2014}{\sA}{Entwurf Use Case Diagramme}
		\vhEntry{1.5}{20.10.2014}{\sA}{Formulierung der nichtfunktionalen Anforderungen}
		\vhEntry{1.6}{21.10.2014}{\sA}{Entwurf Data-Flow-Diagramme}
		\vhEntry{1.7}{10.11.2014}{\sA}{Entwurf Structured Design}
		\vhEntry{1.8}{03.12.2014}{\sA}{Entwurf Modulares Design}
	\end{versionhistory}
	
\clearpage

	\section{Zielbestimmung}
		\subsection{Musskriterien}			

			
			\begin{itemize}
			\item[-]	Verschiedene Benutzergruppen
			\item[-]	Neue Benutzer anlegen und administrieren 
			\item[-]	Rechtehierarchien
			\item[-]	Studenten können ihre Noten in dem System einsehen.
			\item[-]	Notenkonvertierungsprofile erstellen und verwalten
			\item[-]	Dozenten weisen einer Veranstaltung Bewertungsschema zu
			\item[-]	Ein Template kann zu einer Veranstaltung erstellt werden.
			\item[-]	Veranstaltungen des Typs (Gruppenarbeit, Einzelarbeit) können erstellt werden.
			\item[-]	Die Festlegung eines Grundscore (Gruppe, Einzeln) lässt sich in zwei skalierbare \newline Bereiche beliebig teilen ($H_2$)
			\item[-]	Score Refining Faktoren können in einem Template frei eingestellt werden (Items) ($R_2S$)
			\item[-]	Skalierbarkeit des Bewertungssystems von Bewertung der Gruppe zu Einzelperson und umgekehrt.

			\item[-]	Zuweisung von Aufgaben zu einer Studentengruppen
			\item[-]	Veranstaltung durch Dozent oder Prüfer.				

			\item[-]	Zuweisung von beliebig gruppierten Studenten zu einer Veranstaltung durch Dozent oder Prüfer.				

			\end{itemize}

			
		\subsection{Wunschkriterien}
		\begin{itemize}
		\item[-]	Alle Zwischenbewertungen die zur Note führen werden dem Studenten angezeigt.
		\item[-]	Profilverwaltung um standardisierte Bewertungen erzeugen zu können.
		\end{itemize}
		
		\subsection{Abgrenzungskriterien}
		\subsubsection*{Veranstaltungsplanung und Erstellung der Bewertungsroutine}
		
		Ein Dozent bekommt mit dieser Anwendung ein  Baukasten zu Erstellung einer Bewertung zur Lehrveranstaltung. Die Anwendung stellt mehrere mögliche Bewertungsfunktionen, die kombiniert werden können. Durch das kombinieren der Funktionen können Bewertungen beinahe beliebiger Komplexität erstellt werden.
		
		Für die Simulationszwecke beschränkt sich der Prototyp auf die Module ($H_2$ , $R_2S$, $S_2G$)
		
		\subsubsection*{Bewertungslogik}
		
		Die Anwendung unterstützt zunächst die Bewertung der Veranstaltungen die grundsätzlich mit einem bestanden / nicht bestanden, also 0 oder 1 bewertet werden können. Dies wird als Grundlage für alle weiteren Berechnungen als Grundlage genommen. Die Ermittlung einer Note kann sowohl für Einzelperson, als auch für die Mitglieder einer Gruppe durchgeführt werden. 

		\subsubsection*{Veranstaltungsarten}
		
		Jede erstellte Veranstaltung beinhaltet genau ein Bewertungsschema (keine Konsolidierung mehrerer Veranstaltungen möglich) und führt somit zu einer Note. Die Erstellung einer Note aus der Kombination von verschiedenen Einzelnoten ist nicht möglich.

		
	\section{Produkteinsatz}
	
		
		\subsection{Anwendungsbereiche}
		\begin{itemize}
		\item[-]	Planung einer Veranstaltung mit anstehenden Prüfungsleistungen für Lehrveranstaltungsplaner
		\item[-]	Errechnung einer Note durch den Prüfer anhand festgelegter Kriterien.
		\item[-]	Einsicht der Noten durch Studenten, sowie die Leistungsübersicht für die Dozenten und Prüfer.
		\end{itemize}
		
		
		\subsection{Zielgruppen}
		
		\begin{description}[\setlabelphantom{Administrator}]
		\item[Administrator] 	 Administrator sichert und verwaltet das Gesamtsystem (Server, Webserver u.A.). Er hat keine Möglichkeit die Inhalte (z.B. Studentennoten) zu manipulieren. 
		\item[Dozent]	Dozenten benutzen diese Anwendung um eine Veranstaltung zu erstellen und eine von Ihnen bestimmte Bewertungsroutine umzusetzen.
		\item[Prüfer] Prüfer nutzen diese Anwendung um auf eine einfache zentralisierte Art und Weise in der ihnen zugewiesenen Veranstaltung Bewertungen durchzuführen.
		\item[Studenten]	Studenten benutzen diese Anwendung um eine Auskunft über ihre Leistungen in den belegten Veranstaltungen zu erhalten.
		\end{description}

		\subsection{Betriebsbedingungen}
		\begin{itemize}
		\item[-]	Das System soll nur einmal angelegt werden und autark funktionieren.
		\item[-]	Klar definierte Benutzergruppen und Rechte
		\item[-]	Datenbankschnittstelle für die Benutzerverwaltung (mögliche Anbindung an Dualis bei vorhandener API)
		\item[-]	Datenbank-Umgebung für die Speicherung der Daten
		\item[-]	Archivierung der Daten (Datensicherheit wegen Einsehbarkeit der Daten)
		\item[-]	Zugriff von außen: Sowohl Lesen, als auch als Edit möglich. 
		\item[-]	Die Anwendung soll parallelen Zugriff von mehreren Benutzern zulassen.
		\item[-]	Der Administrator kann die Datenbank sichern und die Programmdaten ändern.\\
					Er hat keine Lese-/Schreibrechte auf die Datenbankinhalte.
		\end{itemize}

		
		
	\section{Produktübersicht}
	
	Das Produkt wird durch folgende Use Cases beschrieben. Alle Use Cases beziehen sich auf das Verhalten des Systems aus der Sicht eines außenstehenden Anwenders. Die Use Cases beschreiben die Hauptfunktionen, welche ein Benutzer des Systems wahrnehmen kann.\\
	Die Anforderung an die Funktionalität entsteht aus der Auswertung der Akteur/Produkt- und Akteur/Akteur-Beziehungen im Kapitel~\ref{chap:Prfunk}.
	
	\subsection{Use Case Analyse}
	
	\begin{table}[H]
	\begin{tabular}{|ll}
	 \rowcolor{hellgrau}\textbf{Use Case ID} & 1 \\
 	 \textbf{Elementarer Geschäftsprozess} &  Veranstaltung planen \\ 
	 \textbf{Ziel des Use Cases} & Darstellung des  Prozesses einer    \\
	 & Veranstaltungsplanung\\ 
	 \textbf{Umgebende Systemgrenze}& Veranstaltungskonfigurator \\ 
	 \textbf{Vorbedingung} & Dozent ist angemeldet \\ 
	 \textbf{Nachbedingung Erfolg} & Veranstaltung ist angelegt \\ 
	 & Template ist erstellt\\
	 \textbf{Beteiligte Nutzer:} & Dozent \\ 
     \textbf{Auslösendes Ereignis:} & Dozent möchte Veranstaltung planen \\ 
	 
	\end{tabular} 
	\label{tab:usecase_1}
	\end{table}
	
		\begin{table}[H]
		\begin{tabular}{|ll}
		  \rowcolor{hellgrau}\textbf{Use Case ID} & 2 \\
	 	 \textbf{Elementarer Geschäftsprozess} &  Template erstellen \\ 
		 \textbf{Ziel des Use Cases} & Templates für die Veranstaltungen können \\& angelegt werden   \\
		 \textbf{Umgebende Systemgrenze}& Veranstaltungskonfigurator \\ 
		 \textbf{Vorbedingung} & Dozent ist angemeldet \\ 
		 \textbf{Nachbedingung Erfolg} & Template ist erstellt \\ 
		 \textbf{Beteiligte Nutzer:} & Dozent \\ 
	     \textbf{Auslösendes Ereignis:} & Dozent möchte Template erstellen \\ 
		 
		\end{tabular} 
		\label{tab:usecase_2}
		\end{table}
	
	
			\begin{table}[H]
			\begin{tabular}{|ll}
			  \rowcolor{hellgrau}\textbf{Use Case ID} & 3 \\
		 	 \textbf{Elementarer Geschäftsprozess} &  Bewerten \\ 
			 \textbf{Ziel des Use Cases} & Prüfer bewertet  \\
			 \textbf{Umgebende Systemgrenze} & Bewertung \\ 
			 \textbf{Vorbedingung} & Veranstaltung existiert \\
			  					   & Studenten/Gruppen existieren \\ 
			 \textbf{Nachbedingung Erfolg} & Teilbewertung liegt vor \\
			 							   & Bewertung ist abgeschlossen\\
			 							    
			 \textbf{Beteiligte Nutzer:} & Prüfer \\ 
		     \textbf{Auslösendes Ereignis:} & Prüfer will Gruppe/Student bewerten \\ 
			 
			\end{tabular} 
			\label{tab:usecase_3}
			\end{table}
			
			\begin{table}[H]
			\begin{tabular}{|ll}
			  \rowcolor{hellgrau}\textbf{Use Case ID} & 4 \\
		 	 \textbf{Elementarer Geschäftsprozess} &  Einsehen \\ 
			 \textbf{Ziel des Use Cases} & Studenten sehen ihre Noten ein  \\
			 \textbf{Umgebende Systemgrenze} & Bewertung \\ 
			 \textbf{Vorbedingung} & Bewertung liegt bereits vor \\
			 \textbf{Nachbedingung Erfolg} & Student sieht die Note\\
			 			    
			 \textbf{Beteiligte Nutzer:} & Student \\ 
		     \textbf{Auslösendes Ereignis:} & Student will seine Note sehen\\ 
			 
			\end{tabular} 
			\label{tab:usecase_4}
			\end{table}
			
			\begin{table}[H]
			\begin{tabular}{|ll}
			  \rowcolor{hellgrau}\textbf{Use Case ID} & 5 \\
		 	 \textbf{Elementarer Geschäftsprozess} &  Teilnehmer anlegen \\ 
			 \textbf{Ziel des Use Cases} & Prüfer oder Dozent konfigurieren \\& die Teilnehmer einer Veranstaltung  \\
			 \textbf{Umgebende Systemgrenze} & Planung Prüfung \\ 
			 \textbf{Vorbedingung} & Veranstaltung ist konfiguriert \\
			 \textbf{Nachbedingung Erfolg} & Veranstaltung ist freigegeben\\
			 			    
			 \textbf{Beteiligte Nutzer:} & Prüfer,Dozent \\ 
		     \textbf{Auslösendes Ereignis:} & Studenten sollen einer Veranstaltung hinzugefügt werden\\ 
			 
			\end{tabular} 
			\label{tab:usecase_5}
			\end{table}

						
		\clearpage
		\subsection{Use Cases in grafischer Darstellung}
		
	\begin{figure}[th!]
	\centering
	\includegraphics[width=\textwidth]{./img/use_case}
	\caption{Darstellung des Gesamtsystems anhand der ausgearbeiteten Use-Cases}
	\label{fig:use_case}
	\end{figure}
	\newpage
	
	\section{Produktfunktionen\label{chap:Prfunk}}
	Im folgenden sind die Geschäftsprozesse der Anwendung dokumentiert.\\
	Diese Produktfunktionen werden im Laufe der Entwicklung angepasst und in ihrem Umfang erweitert.
	Die Geschäftsprozesse finden sich als Diagramme im Anhang (Abb. \ref{fig:process1}, \ref{fig:process2}, \ref{fig:process3})
	
		\begin{table}[H]
			\begin{tabular}{|ll}
				\rowcolor{hellgrau}\multicolumn{2}{l}{/\textbf{\textit{F10}}/}\\\hline
				 \textbf{Geschäftsprozess} & Bewertungsschema festlegen \\ 
				 \textbf{Kategorie} & primär \\ 
				 \textbf{Vorbedingung} & 1. Dozent ist angemeldet  \\ 
				 \textbf{Nachbedingung Erfolg\phantom{xxxx}} & Ein Bewertungsschema steht zur Verfügung\\ 
				 & Prüfer erfährt welchen Kurs (bzw.  Student/en) er zu prüfen hat.\\
				 \textbf{Akteure} & Dozent \\ 
				 \textbf{Auslösendes Ereignis} & Dozent legt ein Bewertungsschema an  \\ 
				 \textbf{Unterfunktionen} & /\textbf{\textit{F11}}/ Umrechnungsformel festlegen\\
				 & /\textbf{\textit{F12}}/ Parameter anpassen
				 \end{tabular} 
			\label{tab:F10}
			\end{table}
	
		\begin{table}[H]
		\begin{tabular}{|ll}
			\rowcolor{hellgrau}\multicolumn{2}{l}{/\textbf{\textit{F20}}/}\\\hline
			 \textbf{Geschäftsprozess} & Template erstellen \\ 
			 \textbf{Kategorie} & primär \\ 
			 \textbf{Vorbedingung} & Dozent ist angemeldet\\
			 & ein Bewertungsschema liegt vor \\ 

			 \textbf{Nachbedingung Erfolg} & Template ist
			  zur Veranstaltungsplanung freigegeben  \\ 
			 \textbf{Nachbedingung Fehlschlag} &  \\ 
			 \textbf{Akteure} & Dozent \\ 
			 \textbf{Auslösendes Ereignis} & Dozent legt ein Template an \\
			 & Bei Veranstaltungserstellung wurde kein Template gefunden\\
			 & Bei Veranstaltungserstellung wird ein neues Template erstellt\\
			 \textbf{Unterfunktionen} & /\textbf{\textit{F21}}/ Die Art der Prüfung wird spezifiziert \\
			 &  /\textbf{\textit{F22}}/ Items hinzufügen\\
			 & /\textbf{\textit{F23}}/ Bewertungsschema hinzufügen\\
			  \textbf{Erweiterung} & 1. Templates refactoring \\ 
			 \textbf{Alternativen} & 1. Neues Bewertungsschema erstellen anlegen \\  			
			 \end{tabular} 
		\label{tab:F20}
		\end{table}
	
		
			\begin{table}[H]
			\begin{tabular}{|ll}
				\rowcolor{hellgrau}\multicolumn{2}{l}{/\textbf{\textit{F30}}/}\\\hline
				 \textbf{Geschäftsprozess} & Veranstaltung planen \\ 
				 \textbf{Kategorie} & primär \\ 
				 \textbf{Vorbedingung} & Dozent ist angemeldet\\
				 & ein Template liegt vor \\ 
				  	& Bewertungsschema liegt vor\\
				 	& Prüfer liegen vor\\
				 \textbf{Nachbedingung Erfolg} & Eine Veranstaltung ist angelegt  \\ 
				 & Prüfer erfährt welchen Kurs (bzw.  Student/en) er zu prüfen hat.\\
				 \textbf{Nachbedingung Fehlschlag} &  \\ 
				 \textbf{Akteure} & Dozent \\ 
				 \textbf{Auslösendes Ereignis} & Dozent legt eine Veranstaltung an  \\ 
				 \textbf{Unterfunktionen} & /\textbf{\textit{F31}}/ Dozent konfiguriert das Template\\ 
				 & /\textbf{\textit{F32}}/ Dozent befüllt(prüft) die Parameter des Bewertungsschemas\\
				 &/\textbf{\textit{F33}}/ Dozent kann Prüfer einer Veranstaltung zuteilen zuteilen \\
				 & /\textbf{\textit{F34}}/ Dozent kann Studenten/Gruppen der  Veranstaltung zuweisen\\
				  \textbf{Erweiterung} &  \\ 
				 \textbf{Alternativen} & 1. Neues Template anlegen \\
				 & 2. Neue Studenten anlegen\\
				 & 3. Neue Prüfer Anlegen \\
				 & 4. Neues Bewertungsschema anlegen
				 \end{tabular} 
			\label{tab:F30}
			\end{table}
	
	
			\begin{table}[H]
			\begin{tabular}{|ll}
				\rowcolor{hellgrau}\multicolumn{2}{l}{/\textbf{\textit{F40}}/}\\\hline
				 \textbf{Geschäftsprozess} & Bewerten \\ 
				 \textbf{Kategorie} & primär \\ 
				 \textbf{Vorbedingung} & 1. Dozent/Prüfer ist angemeldet  \\ 
				 & 2. Veranstaltung ist angelegt\\
				 & 3. Gruppen/Student ist einer Veranstaltung zugewiesen\\
				 & 4. Prüfer ist einer Veranstaltung zugewiesen\\
				 \textbf{Nachbedingung Erfolg\phantom{xxxx}} & Ein Student/Gruppe wurde bewertet\\
				 & Teilbewertung/vollständige Bewertung ist abgeschlossen\\ 
				 \textbf{Akteure} & Dozent \\ 
				 & Prüfer\\
				 \textbf{Auslösendes Ereignis} & Prüfer will bewerten \\ 
				 \textbf{Unterfunktionen} &  /\textbf{\textit{F41}}/ Prüfer wählt Veranstaltung aus\\
				 & /\textbf{\textit{F42}}/ Prüfer wählt Student/Gruppe zur Bewertung aus\\
				 &/\textbf{\textit{F43}}/  Bewertungen/Teilbewertungen eintragen\\
				 & /\textbf{\textit{F44}}/ Bewertung abschließen\\
			
				 \end{tabular} 
			\label{tab:F40}
			\end{table}
			
\begin{table}[H]
	\begin{tabular}{|ll}
		\rowcolor{hellgrau}\multicolumn{2}{l}{/\textbf{\textit{F50}}/}\\\hline
		 \textbf{Geschäftsprozess} & Bewertung abschließen\\ 
		 \textbf{Kategorie} & primär \\ 
		 \textbf{Vorbedingung} & Bewertungen einer Veranstaltung in allen Items liegen vor\\
		 & Bewertungen für einen Student/Gruppe sind komplett\\
		 & Prüfer ist angemeldet \\ 
		 \textbf{Nachbedingung Erfolg} & Dozent erfährt, dass die Endnote der Veranstaltung vorliegt  \\ 
		 & Prüfer verliert das Recht zu editieren, nur noch sehen\\
		 & Nur Dozent ist berechtigt zu ändern\\
		 \textbf{Nachbedingung Fehlschlag} &  Die Bewertung steht weiter aus\\
		 & Bestätigung für den Abschluss trotz der \\
		 & unvollständigen Daten wird erfragt\\ 
		 \textbf{Akteure} & Dozent \\ 
		 & Prüfer\\
		 \textbf{Auslösendes Ereignis} & Prüfer schließt die Bewertung ab \\ 
		 \textbf{Alternativen} & 1. Bewertung vervollständigen \\
		 \end{tabular} 
	\label{tab:F50}
	\end{table}		
			
	\begin{table}[H]
		\begin{tabular}{|ll}
			\rowcolor{hellgrau}\multicolumn{2}{l}{/\textbf{\textit{F60}}/}\\\hline
			 \textbf{Geschäftsprozess} & Kurs anlegen\\ 
			 \textbf{Kategorie} & primär \\ 
			 \textbf{Vorbedingung} & 1. Dozent ist angemeldet \phantom{aaaaaaaaaaaaaaaaaaaaaaaaaaaaaaa} \\
			  & Studenten liegen vor\\
			 \textbf{Nachbedingung Erfolg} & Kurs ist angelegt\\
			 \textbf{Nachbedingung Fehlschlag} & -\\
			 \textbf{Akteure} & Dozent \\ 
			 \textbf{Auslösendes Ereignis} & Dozent legt einen Kurs an\\ 
			 \textbf{Unterfunktionen} &  /\textbf{\textit{F61}}/  Kurs erstellen\\
			 & /\textbf{\textit{F62}}/  Studenten hinzufügen\\
			 & /\textbf{\textit{F63}}/  Kurs erstellen\\
			 \textbf{Alternativen} & 1. Studenten anlegen \\
			 \end{tabular} 
		\label{tab:F60}
		\end{table}
		
		
			\begin{table}[H]
				\begin{tabular}{|ll}
					\rowcolor{hellgrau}\multicolumn{2}{l}{/\textbf{\textit{F70}}/}\\\hline
					 \textbf{Geschäftsprozess} & Studenten anlegen\\ 
					 \textbf{Kategorie} & primär \\ 
					 \textbf{Vorbedingung} & 1. Dozent ist angemeldet \phantom{aaaaaaaaaaaaaaaaaaaaaaaaaaaaaaa} \\
					 \textbf{Nachbedingung Erfolg} & Ein Student ist angelegt\\
					 & Student kann einem Kurs/Gruppe hinzugefügt werden\\
					 \textbf{Nachbedingung Fehlschlag} & -\\
					 \textbf{Akteure} & Dozent \\ 
					 \textbf{Auslösendes Ereignis} & Dozent legt Studenten an\\ 
					 \textbf{Unterfunktionen} &  /\textbf{\textit{F71}}/  Studentenregisterkarte erstellen\\
					 & /\textbf{\textit{F72}}/ Daten befüllen\\
				 \end{tabular} 
				\label{tab:F70}
				\end{table}	
			
			\begin{table}[H]
				\begin{tabular}{|ll}
				\rowcolor{hellgrau}	\multicolumn{2}{l}{/\textbf{\textit{F80}}/}\\\hline
					 \textbf{Geschäftsprozess} & Bewertung einsehen\\ 
					 \textbf{Kategorie} & primär \\ 
					 \textbf{Vorbedingung} & 1. Student ist angemeldet \phantom{aaaaaaaaaaaaaaaaaaaaaaaaaaaaaaa} \\
					 &2. Bewertung liegt vor\\
					 \textbf{Nachbedingung Erfolg} & 1. Student sieht seine Noten\\
					  \textbf{Nachbedingung Fehlschlag} & Benachrichtigung\\
					 \textbf{Akteure} & Student \\ 
					 \textbf{Auslösendes Ereignis} & Student will seine Noten einsehen\\ 
					 \textbf{Unterfunktionen} &  /\textbf{\textit{F81}}/  Student meldet sich an\\
					 & /\textbf{\textit{F82}}/  Student sieht seine Noten\\
				 \end{tabular} 
				\label{tab:F80}
				\end{table}	


	
	
	\section{Produktdaten}
	
		Die Größe der einzelnen Datenpunkte richtet sich variabel nach der letztendlichen Größe des Anwendungssystems. Die praktische Obergrenze wird durch die Hardware bzw. Datenbank festgelegt. 
		
		Die unten aufgeführten Produktdaten in der Notation /Dxx/ beschreiben die Gesamtheit der Daten, welche die Anwendung benötigt oder erstellt. Das vorläufige Datenbankmodell ist in der Abbildung~\ref{fig:ERM} dargestellt, erst in dem ER-Modell werden die Trennungen in die Tabellen vorgenommen.
		
	 
		
	\begin{description}[\setlabelphantom{xxxxxx}]
	\item[/D10/] Benutzer \{Benutzer\_id, Benutzertyp(Rechte) , Personendaten, Veranstaltungen\}\\ (Max. 100000) 
	\item[/D20/] Veranstaltung \{Veranstaltungs\_id, Bewertungsschema(S$_2$G), Bewertungstemplate, Prüfer, Stundenten, Score, Note, Status\_Bewertung\, Beschreibung\} \\ (Max. 100000)
	\item[/D30/] Serialisierte Objekte
	\item[/D40/] Datenarchiv (Alle Daten)
	\end{description}
	
	
	\begin{figure}[H]
\centering
\includegraphics[width=0.7\textwidth]{./img/ERM}
\caption{Entity Relation Modell in IDEF1X Notation}
\label{fig:ERM}
\end{figure}

	
	
	\section{Produktleistungen}
	
	\begin{description}[\setlabelphantom{xxxxxxx}]
	\item[/L10/] Die Antwortzeiten der Anwendung sollen so gering wie möglich gehalten werden.
	\item[/L20/] 	Die steigenden Benutzerzahlen die zur gleichen Zeit auf das System zugreifen, sollen keine signifikanten Einflüsse auf die Antwortzeiten des Systems haben.
	\item[/L30/]	Benutzerfreundliche UX soll implementiert werden.
	\item[/L40/]	Die Anwendung besitzt eine eigene Datenbank zur Speicherung und Verwaltung der Anwendungsdaten.
	Die Ausgliederung einzelner Teilmodule, wie z.B. die Benutzerverwaltung ist möglich (API muss von Kunden bereitgestellt werden).
	\item[/L50/] 	Die Abfragen und die Einsicht der Daten von Außerhalb (WWW) ist entsprechend der Benutzerberechtigung möglich. (wird im Prototypen zu einem späteren Zeitpunkt realisiert)
	\item[/L60/] 	Datenbeständigkeit soll garantiert werden.
	\end{description}


	
	\section{Qualitätsanforderungen}
	\begin{table}[ht]
	\caption{Qualitätsmerkmale nach DIN ISO 9126 – siehe Anhang~\ref{sec:quali} in T3-4}
	\centering
		\begin{tabular}{|c|c|c|c|c|}
		\hline Produktqualität & sehr gut & gut & normal & nicht relevant \\ 
		\hline \multicolumn{5}{|c|}{Funktionalität}   \\ 
		\hline Angemessenheit &  &  & \checkmark &  \\ 
		\hline Richtigkeit &  & \checkmark  &  &  \\ 
		\hline Interoperabilität &  &  &\checkmark  &  \\ 
		\hline Ordnungsmäßigkeit &  &  &\checkmark  &  \\ 
		\hline Sicherheit & & \checkmark & &  \\ 
		\hline \multicolumn{5}{|c|}{Zuverlässigkeit}   \\ 
		\hline Reife &  &   & \checkmark &  \\ 
		\hline Fehlertoleranz &  &  \checkmark &  &  \\ 
		\hline Wiederherstellbarkeit & \checkmark  &  &  &  \\ 
		\hline \multicolumn{5}{|c|}{Benutzbarkeit}  \\ 
		\hline Verständlichkeit &  &  & \checkmark  &  \\ 
		\hline Erlernbarkeit &  &   &  &\checkmark  \\ 
		\hline Bedienbarkeit &  &   & \checkmark &  \\ 
		\hline \multicolumn{5}{|c|}{Effizienz} \\ 
		\hline Zeitverhalten &  &  &\checkmark   &  \\ 
		\hline Verbrauchsverhalten &  &  &\checkmark   &  \\ 
		\hline \multicolumn{5}{|c|}{Änderbarkeit} \\ 
		\hline Analysierbarkeit &  &   & \checkmark &  \\ 
		\hline Modifizierbarkeit &  &  &  \checkmark  &  \\ 
		\hline Stabilität &  &  & \checkmark &  \\ 
		\hline Prüfbarkeit &   &  &\checkmark  &  \\ 
		\hline \multicolumn{5}{|c|}{Übertragbarkeit}  \\ 
		\hline Anpassbarkeit &  &  &\checkmark  &  \\ 
		\hline Installierbarkeit &  &  &  \checkmark &  \\ 
		\hline Konformität	 &  &  &\checkmark   &  \\ 
		\hline Austauschbarkeit &  &  &   & \checkmark \\ 
		\hline 
		\end{tabular} 
	\label{tab:quali_anf}
	\end{table}
	
\clearpage
		\begin{description}[\setlabelphantom{Wiederherstellbarkeit}]
			\item[Richtigkeit]  Da das System mit persönlichen und sensiblen Daten arbeitet, sollte dieses eine hohe Richtigkeit haben. Die Tests auf die Richtigkeit und die Robustheit des Systems sind mit dem Entwicklungsprozess verflochten und basieren auf den vorgestellten Benutzerszenarien (Agile Development). 
			\item[Sicherheit] Die Sicherheitsaspekte der einzelnen Module werde im Lauf der Entwicklung integriert.
			Im Prototyp wird für das Anmeldeverfahren per Passwort der Sicherheitsstandard PKCS\#5 (PBKDF2, RFC2898) verwendet und die Sichten der einzelnen Bereiche der Anwendung strikt an die Benutzerberechtigung gebunden. 
			Für die Umsetzung der Anwendung soll eine gesicherte Verbindung über HTTPS (SSH, RS2818) bereit gestellt werden.
			\item[Fehlertoleranz]  Bei der Berechnung der Noten sollten so wenig Fehler wie möglich auftreten 
			\item[Wiederherstellbarkeit]  Die Daten des Systems müssen entsprechend der jeweiligen Anforderung der Hochschule über einen längeren Zeitraum einsehbar und wiederherstellbar sein.
			Da alle benutzerbezogenen Daten in der Datenbank gespeichert und verwaltet werden bezieht sich die Wiederherstellbarkeit auf der MYSQL-Standard.
			Die Entwicklung der Backupstrategie folgt im weiteren Verlauf der Entwicklung.
		\end{description}	
	

	\section{Benutzeroberfläche}
	
	Die Benutzeroberfläche [UI] (\underline{\textbf{U}}ser \underline{\textbf{I}}nterface) entspricht den modernen Ansprüchen einer Web-Anwendungen und wird mit Google Designrichtlinien erstellt. 
	
 	Das Leitmotiv für die Entwicklung des UI  ist "\textit{Focus on the user and all else will follow}" (engl.: konzentriere dich auf den Bentzer und der Rest folgt.).
 	
 	 \url{http://www.google.com/design/}
	
	Die genauere Spezifikation der grafischen Benutzeroberfläche wird zu einem späteren Zeitpunkt des Projektes definiert. 
	
	\begin{table}[H]
	\centering
	\caption{Benutzergruppen und Rechteverteilung}
	\begin{tabular}{|c|c|c|c|c|}
	\hline Benutzergruppe & Lesen & Schreiben & Ändern & Systemanpassungen \\ 
	\hline Administrator & \checkmark & \checkmark & \checkmark & \checkmark \\ 
	\hline Verwaltung & \checkmark & \checkmark & \checkmark &  \\ 
	\hline Prüfer & \checkmark & \checkmark &  &  \\ 
	\hline Student & \checkmark &  &  &  \\ 
	\hline 
	\end{tabular} 
	\label{tab:usergroup}
	\end{table}
	
	\paragraph{Benutzersichten}
	
	Jede Benutzergruppe erhält eine eigene Anwengungssicht. Die jeweiligen Sichten bieten unterschiedliche Möglichkeiten zur Interaktion entsprechend der in Kapitel~\ref{chap:Prfunk} definierten Funktionen.
	

	
	\section{Nichtfunktionale Anforderungen}
	
	\begin{description}[\setlabelphantom{/QEZ00/}]
	\item[/Q10/] Das System muss die Rollen, wie in der Tabelle~\ref{tab:usergroup} dargestellt, unterscheiden und die Rechteverteilung sicherstellen können.
	\item[/Q20/] Für die im System gespeicherten Daten muss gewährleistet werden, dass deren Sicherung zuverlässig passiert.
	\end{description} 

	
	
	
	\section{Technische Produktumgebung}
	
	Das System läuft auf einem Server, der die Anfragen der webbasierten Clients entgegennehmen kann.\\
	Die GUI für den Web-Browser wird erst im nächsten Entwicklungsstadium bereitgestellt.
	
		\subsection{Software}
		Server-Betriebssystem: UNIX\\
		Clientsysteme: Browser (Versionen ab 2013)\\
		Web-Server: Jetty (Kontext für Java Server Faces)
		Datenbank: MySQL (Anbindung durch JDBC)
		
		\subsection{Hardware}
		Leistungsstarker Server\\
		Für den Prototypen ist nur eine eingeschränkte Funktionalität vorgesehen
		
		\subsection{Orgware}
		
		Zunächst ist das System nur durch freigegebene Benutzer erreichbar. 		

		\subsection{Produkt-Schnittstellen}
		
		Web-Schnittstelle, Serverschnittstellen
		
	\section{Spezielle Anforderungen an die Entwicklungs-Umgebung}
		\subsection{Software}
		Git Versionsverwaltung, Issue Tracking System - YouTrack, IDE für JAVA Development, Database Development Tools
		
		\subsection{Hardware}
		Laptop, Standalone PC (Server, Client)
		\subsection{Orgware}
		XP (extreme proramming und agile development)	
		\subsection{Entwicklungs-Schnittstellen}
		WEB-Schnittstelle
		
	\section{Gliederung in Teilprodukte}
	
	Im Laufe der Produktentwicklung werden folgende Produkte an den Kunden ausgeliefert zum Zweck des Tests und der Produktivsetzung:
	
	\subsection{Prototyp ohne UI}
	Der Prototyp besteht aus einem  Veranstaltungskonfigurator, einem Bewertungssystem und einem Modul zum Einsehen der Bewertung. Ein UI wird nur zu Testzwecken angebunden um die Schnittstelle zu simulieren und zu prüfen. \\
	Veranstaltungskonfigurator beschränkt sich hierbei auf einen minimalen Funktionsumfang um die mögliche Arbeitsweise deutlich zu machen.\\ 
	Konfigurieren bedeutet in diesem Zusammenhang die Notenumrechnung mit Parametern zu belegen und zwischen unterschiedlichen Bewertungsschemata auswählen zu können.\\ 
	Des Weiteren lassen sich zu einer Veranstaltung einzeln gewichtete Bewertungskriterien hinzufügen.\\ 
	Das Bewertungssystem beinhaltet eine vom Dozent vordefinierte Maske, in die der Prüfer die Ergebnisse der Veranstaltung eintragen kann. Anhand der Ergebnisse wird dann eine Note für diese Veranstaltung errechnet.
	Über ein Modul zum Einsehen der Bewertung, können zum einen Dozent und Prüfer die eingegeben Ergebnisse überprüfen, zum anderen die Studenten ihr Abschneiden bei der jeweiligen Veranstaltung ansehen.
	
	\subsection{Erweiterter Simulationsprotyp mit UI}
	Der Prototyp wird modular erweitert und mit weiterer Funktionalität belegt bis alle Anforderungen erfüllt sind. In diesem Stadium hat die Anwendung den vollen Funktionsumfang. Die Anwendung kann bereits zum Test der Umgebung und der Hardware eingesetzt werden. 
	In diesem Stadium der Entwicklung besteht kein Ansprüche auf die fehlerfreie Funktion der Anwendung.
	
	\subsection{Produktive Anwendung}
	In diesem Stadium werden alle Systemanpassungen durchgeführt und die Anwendung läuft fehlerfrei. Auf Anfrage können weitere Funktionen über den Erweiterten Simulationsprototypen eingebunden werden. Dadurch entstehen weitere Kosten.
	
	\subsection{Sicherheitskonzept}
	Die Standards in Sicherheit der Verbindung und Datenübertragung werden in der Anwendung, als Grundfunktionalität vorgesehen und ist in den Entwicklungsprozess integriert.
	

	Die Ausarbeitung eines erweiterten Sicherheitskonzepts für die Daten und Zugriffe, sowie gegen Angriffe wird als Teilprodukt angeboten. Dabei stehen dem Kunden drei Optionen zur Auswahl: Schwach, Normal, Hoch. 
	
	Die Einstufung orientiert sich vor allem an der Komplexität des Konzeptes und dem Umfang der Tests. Daraus entstehen zusätzliche Arbeitspakete, welche separat abgerechnet werden. 
	

	Die Ausarbeitung eines Sicherheitskonzepts für die Daten wird als Teilprodukt angeboten. Dabei stehen dem Kunden drei Optionen zur Auswahl: Schwach, Normal, Hoch. 
	Die Einstufung orientiert sich vor allem an der Komplexität des Konzeptes und dem Umfang der Tests. \newline	
	Unabhängig vom gewählten Modell werden  Grundsicherheitsstandards implementiert.

	\subsection{Back-Up Planung}
	
	Die Back-Up Planung läuft über Standardmodelle der Datenbank. Für erweiterte Back-Up Strategien muss ein gesonderter Auftrag erstellt werden. Auf Wunsch des Kunden sind unterschiedliche Konzepte umsetzbar.
	
	\subsection{User Experience und User Interface}
	
	Die Anwendung ist mit einer herkömmlichen Benutzeroberfläche ausgestattet. Diese wird parallel zum Prototypen entwickelt. Eine Erweiterung und die Anpassung der Benutzeroberfläche und vor allem der User Experience resultiert in einem extra Arbeitspaket.
	
	
	\subsection{Maintenance}
	
	Die Ausarbeitung eines Instandhaltungsonzepts ist nicht Teil des angebotenen Umfangs, kann jedoch auf Wunsch des Kunden in einem extra Arbeitspaket umgesetzt werden.
	
	\section{Ergänzungen}
		
	
	
 \begin{appendix}
  \section{Begriffsdefinitionen}
	\begin{description}[\setlabelphantom{Notenkonvertierungsprofilxx}]
	\item[Benutzergruppen] Mögliche Benutzer werden mit verschiedenen Rechten und Bedienszenarien versehen.
	\item[Student] Eine Benutzergruppe mit stark eingeschränkten Zugriffsrechten. Kann nur die vom Dozent freigegebenen Informationen sehen.
	\item[Dozent] Eine Benutzergruppe mit umfassenden Rechten. Kann Templates, Veranstaltungen und Benutzergruppen erstellen und verwalten.
	\item[Prüfer] Ein Benutzer mit Rechten zur Dateneingabe. Kann nichts an einer Veranstaltung ändern. Muss sich an die vom Dozent vorgegeben Muster halten.
	\item[Sekretariat] Benutzergruppe, welche die Rechte zur Datenverwaltung besitzt (kein Löschrecht).
	\item[Administrator] Eine Benutzergruppe mit der Berechtigung zur systemnahen Anpassungen.
	\item[Veranstaltung] Template + Bewertungsschema + angepasste Items + Studenten/Gruppen + Kurs + Prüfer . Eine Veranstaltung umfasst die obengenannte Bausteine und wird vom Dozenten freigegeben. 
	\item[Template] Ein Template enthält eine beliebige Anzahl von Items und evtl. ein oder mehrere zugewiesene Bewertungsschemata. Ein Template darf leer sein. Somit werden komplett neue Veranstaltungen planbar.
	\item[Item] Ein entweder fest oder variabel angelegtes Bewertungskriterium. Diese Kriterien können für eine Veranstaltung angepasst werden.
	\item[{\parbox[t]{0.2\linewidth}{Notenkonvertierungsprofil \\ (S$_2$G)}}] {\parbox[t]{\linewidth}{ Dies ist eine Umrechnungsformel, die sich aus der Skala, wie der Score in eine Note umgerechnet werden soll, bestimmt. Das konkrete Profil kann für eine Veranstaltung durch Eingangsparameter angepasst werden.}}
	\item[Score] Punkte die eine Basis für die Notenbildung darstellen.
	\item[Grundscore (H$_2$)]  Zweistufige hollistische Bewertung. Diese zwei Bereiche sind frei einstellbar.
	\item[Score Refining (S$_2$R)]  Ein Schema zur Aufbereitung des Grundscore durch Faktoren (refining). Die Faktoren werden vom Dozenten vordefiniert.
	
	 
	\end{description}
  \section{Abkürzungen}
  
  	\begin{description}[\setlabelphantom{xxxxxx}]
  	\item[$API$] Application Programming Interface. Externe Programmschnittstelle, hier für Anbindung an externe Datenbank.
  	\item[$S_2G$] Punkte zu Note (score to grade) Konvertierung  anhand festgelegter Formel und Parameter.
  	\item[$R_2S$] Wertung zu Punkten (rate to score). Punkte werden anhand manuell eingegebener Rates errechnet. 
  	\item[$H_2$] Hollistisches Bewertungsmodul. Zwei Bereiche (frei konfigurierbar). Ausgang ist eine Score, welche weiter in $R_2S$.
  	
  	
  	\end{description}
  \section{Modelle}
  \subsection{Geschäftsprozesse}
  

  	Aus den Use Cases in Abbildung 1 ergeben sich die folgende Geschäftsprozesse.\\
  	Die Diagramme beschreiben systeminterne Abläufe. Die im folgenden benutzte Nomenklatur entspricht der EPK (Ereignisgesteuerte Prozesskette) Syntax.
  	
  	\begin{figure}[th!]
  	\centering
  	\includegraphics[width=\textwidth]{./img/EPK_legend}
  	\label{fig:legend}
  	\end{figure}
  	  	
  	\begin{figure}[th!]
  	\centering
  	\includegraphics[width=\textwidth]{./img/EPK_Veranstaltung}
  	\caption{Prozessdiagramm für den Geschäftsprozess Veranstaltungserstellung}
  	\label{fig:process1}
  	\end{figure}
  		
  	\begin{figure}[th!]
  	\centering
  	\includegraphics[width=0.6\textwidth]{./img/EPK_Bewertung}
  	\caption{Ablaufdiagramm für den Geschäftsprozess Bewertung}
  	\label{fig:process2}
  	\end{figure}
  	
  	\begin{figure}[th!]
  	\centering
  	\includegraphics[width=0.6\textwidth]{./img/EPK_Einsehen}
  	\caption{Ablaufdiagramm für den Geschäftsprozess Einsehen der Noten}
  	\label{fig:process3}
  	\end{figure}
  
\clearpage
\subsection{Data Flow Diagramm}
  Die Abbildung~\ref{fig:kontextdiagramm} stellt den groben Datenfluss für die gesamte Anwendung dar:
 
 
   \begin{figure}[ht]
 \centering
 \includegraphics[width=\textwidth]{./img/dfd_context}
 \caption{Kontextdiagramm der Software}
 \label{fig:kontextdiagramm}
 \end{figure}
 Der Anmeldeprozess wurde dabei separiert in Abbildung \ref{fig:login} dargestellt.
 
 \clearpage
 Das Diagramm \ref{fig:Diagramm0} beschriebt den Datenfluss des Bewertungssystems.
 
   \begin{figure}[H]
  \centering
  \includegraphics[width=\textwidth]{./img/dfd_diagramm0}
  \caption{Bewertungssystem}
  \label{fig:Diagramm0}
  \end{figure}
  \clearpage
  
  Durch das Diagramm \ref{fig:Diagramm1} wird der Prozess 'D1-Veranstaltung planen' genauer beschrieben.
   \begin{figure}[H]
   \centering
   \includegraphics[width=\textwidth]{./img/dfd_diagramm1}
   \caption{Veranstaltung planen}
   \label{fig:Diagramm1}
   \end{figure}
	\clearpage
	
	In der Abbildung \ref{fig:Diagramm2} ist der Datenfluss bei einem Bewertungsvorgang abgebildet.
	\begin{figure}[H]
   \centering
   \includegraphics[width=\textwidth]{./img/dfd_diagramm2}
   \caption{Bewerten}
   \label{fig:Diagramm2}
   \end{figure}
 

  \clearpage
 \subsection{Structured Analysis} 
 
 Ausgehend von Abbildung~\ref{fig:kontextdiagramm} kann die Struktur des Datenflusses folgendermaßen spezifiziert werden. Dabei werden die Datenbewegungen innerhalb der verschiedenen Softwareschichten analysiert.
 In der Abbildung~\ref{fig:login} wird der Login-Prozess abgebildet.   
 
   \begin{figure}[ht]
  \centering
  \includegraphics[width=0.8\textwidth]{./img/dfd_login}
  \caption{Datenfluss bei einem Login Prozess}
  \label{fig:login}
  \end{figure}
  
  
  In der Abbildung~\ref{fig:config} ist der Datenfluss bei der Konfigurierung des Templates, eines Kurses oder einer Veranstaltung dargestellt.
  
      \begin{figure}[ht]
    \centering
    \includegraphics[width=0.8\textwidth]{./img/dfd_config}
    \caption{Datenfluss bei einem Konfigurierung-Prozess}
    \label{fig:config}
    \end{figure}
    \clearpage
    
	
	\subsection{Strukturierter Entwurf} 
	\subsubsection{Login}
	Der Login-Prozess ist die erste Sicht die jeder Benutzer unabhängig vom Typ zu sehen bekommt. Die eingegebenen Daten werden gehashed und mit der Datenbank verglichen. Fehlen die Daten, werden dem Benutzer weitere Versuche gewährt. Nach einer festgelegten Anzahl der Versuche wird ggf. der Benutzername bis auf weiteres gesperrt.
	
	Im Falle der Fehler erhält der Benutzer die Auskunft über Fehlerspezifische Nachrichten.
	
	\begin{figure}[H]
		\centering
		\includegraphics[width=0.8\textwidth]{./img/sd_login_process}
		\caption{Strukturdiagramm eines Login-Ptozesses}
		\label{fig:sd_login_process}
	\end{figure}
	
	\clearpage
	\subsubsection{Veranstaltung Planen}
	Die Veranstaltungsplanung erfolgt in drei Schritten. Template wählen oder erstellen, Studenten (Kurs) hinzufügen und Veranstaltungsfreigabe. Templates können aus unterschiedlichen Modulen aufgebaut werden. Beispielhaft werde $H_2$, Items, $S_2G$ Module in der Abbildung~\ref{fig:sd_veranst_planen} dargestellt. 
	
	Nach der Festlegung des Templates werden die entsprechenden Teilnehmer hinzugefügt, also Studenten und die Prüfer. Nach diesem Schritt kann die Veranstaltung freigegeben werden. Mit der Freigabe erfolgt die Versendung der Email an die Studenten und die Prüfer über die bevorstehende Veranstaltung.

		\begin{figure}[H]
			\centering
			\includegraphics[width=\textwidth]{./img/sd_veranst_planen}
			\caption{Strukturdiagramm eines Vorgangs der Veranstaltungsplanung}
			\label{fig:sd_veranst_planen}
		\end{figure}
	\clearpage
	\subsubsection{Bewerten}
	
		Der Bewertungsvorgang besteht daraus, dass zunächst eine zu bewertende  Veranstaltung ausgewählt werden soll. Anschließend werden die Bewertungen für einen Teilnehmer dieser Veranstaltung eingetragen. Ist die Eingabe aller Daten erfolgreich, folgt die Notenkonvertierung und der Teilnehmer erhält seine Note. Diese wird Freigegeben und abgespeichert. Ab hier hat der Prüfer keinen Zugriff auf die Note mehr.
	
			\begin{figure}[H]
				\centering
				\includegraphics[width=0.7\textwidth]{./img/sd_bewerten}
				\caption{Strukturdiagramm eines Vorgangs der Bewertung}
				\label{fig:sd_bewerten}
			\end{figure}
	
	\subsubsection{Noteneinsicht}
	
	Die Noteneinsicht für einen Studenten erfolgt automatisch nach der Anmeldung. Entsprechend der Studentenkennung werden die Daten geladen und eine Sicht der Noten und Nachrichten erzeugt. Mit der Erweiterung der Funktionalität kann sich dieses Diagramm entsprechend vergrößern.
	
				\begin{figure}[H]
					\centering
					\includegraphics[width=0.5\textwidth]{./img/sd_noten_einsicht}
					\caption{Strukturdiagramm eines Vorgangs der Noteneinsicht}
					\label{fig:sd_noten_einsicht}
				\end{figure}
	
\clearpage
	
	\subsection{Modularer Entwurf}
	

	\subsubsection{Veranstaltung erstellen}
		Das Modul \textbf{Veranstaltung erstellen} besteht aus drei Submodulen \textbf{Template erstellen}, \textbf{Studenten/Kurs}, \textbf{Prüfer}. 
		
		Für die Erklärung von \textbf{Template erstellen} siehe Kapitel~\ref{sec:Template}.
		
		Die Untermodule \textbf{Studenten/Kurs}, \textbf{Prüfer} greifen auf die Datenbank zu und holen die Daten für die entsprechende Personen oder Personengruppen. Die Verwaltung erfolgt von extern.
		
		
		
		\begin{figure}[H]
			\centering
			\includegraphics[width=0.7\textwidth]{./img/MD_VeranstaltungPlanen_main}
			\caption{Modulansicht von Modul Veranstaltung erstellen}
			\label{fig:md_veranstaltung_planen}
		\end{figure}
	
	\subsubsection{Template erstellen \label{sec:Template}}
		  Das Modul  besteht\textbf{Template erstellen} aus zwei Submodulen \textbf{Bewertungsschemata} und \textbf{Bewertungsmodul}. Die beiden Submodule operieren auf komplexen Datentypen und sind skalierbar. Das bedeutet, dass im Rahmen der Template-Erstellung die Submodule geschachtelt werden können.  
		
		\textbf{Bewertungsschemata} Modul übernimmt die Verantwortung über die Verwaltung aller Umrechnungsroutinen, die aus einer Score eine Note generieren können.	
		\textbf{Bewertungsmodul} Modul übernimmt die Verwaltung beliebiger Bewertungsmuster. Dieses Modul 	besteht aus Submodulen als Datentyp und kann in seiner Struktur aus der Abbildung~\ref{fig:sd_veranst_planen} nachvollzogen werden.
		\begin{figure}[H]
			\centering
			\includegraphics[width=0.7\textwidth]{./img/MD_VeranstaltungPlanen}
			\caption{Modulansicht von Template erstellen}
			\label{fig:md_template_erstellen}
		\end{figure}
		
	\subsubsection{Bewerten}
		Das Modul \textbf{Bewerten} besteht aus zwei Submodulen \textbf{Veranstaltung holen} und \textbf{Note zuweisen}.
				
		\begin{figure}[H]
			\centering
			\includegraphics[width=0.7\textwidth]{./img/MD_bewerten}
			\caption{Modulansicht von Modul Bewerten}
			\label{fig:md_bewerten}
		\end{figure}
		
		
	\subsubsection{Login}
	Das Modul Login ist ein simples Modul ohne komplexe Abhängigkeiten und Submodule. Das übernimmt die Kommunikation mit der Datenbank und stellt die Richtigkeit der Anmeldedaten fest.
	
	\subsubsection{Noteneinsicht}
	Das Modul Noteneinsicht  greift auf die Datenbank zu und zeigt einem Studenten seine Daten an. Sind die Noten vorhanden, so sieht er seine Leistungen.
	
	\subsubsection{Sichterstellung}
	
	Das Modul Sichterstellung überprüft den Benutzertyp und erstellt ein Typ-gebundene Ansicht der Gesamtapplikation.
		
  \section{Qualitätsmerkmale\label{sec:quali}}
  \clearpage
  
  
  \section{Aufwandsabschätzung}
  
   Die Function Points (FP) setzen sich zusammen aus den Funktionen im Kapitel~\ref{chap:Prfunk}. Die Unterfunktionen in der Summe bilden die einzelnen Einheiten in der Spalte Anzahl. 
  
  	\begin{figure}[ht]
	\centering
	\includegraphics[width=0.87\textwidth]{./files/Aufwand}
	\label{fig:Aufwand}
	\end{figure}

  \end{appendix}

			  
\end{document}